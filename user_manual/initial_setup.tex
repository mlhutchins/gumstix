\documentclass[12pt, letterpaper, onecolumn, oneside]{article}
\usepackage{amsmath, graphicx, natbib}
\linespread{1} %Line spacing
\oddsidemargin -0.5in %Creates 1 inch margins all around
\textwidth 7.5in
\topmargin 0in
\headheight 0in
\headsep 0in
\textheight 9in
\begin{document}

\begin{centering}
\section{WWLLN Service Unit v4 Initial Setup}

\end{centering}

\subsection*{Method 1: SSH Setup}
The SSH setup method requires:
\begin{itemize}
\item{Ethernet cable}
\item{SSH capable computer}
\end{itemize}

\begin{enumerate}
\item{Connect SU to a host computer directly with an ethernet cable}
\item{Set host computer ethernet network settings to:
\begin{verbatim}
address:	192.168.10.1
gateway:	192.168.10.100
netmask:	255.255.255.0
\end{verbatim}}
\item{SSH into the SU from host computer:
\begin{verbatim}
ssh -p 7777 sferix@192.168.10.2
password: [	]
\end{verbatim}}
\item{Set desired static ip configuration in file $\sim$/networkSetup.sh}
\item{\begin{verbatim}
sudo ./networkSetup.sh
\end{verbatim}}
\item{Switch SU to main network ethernet within 1 minute of running networkSetup.sh}
\item{Test connection by SSH'ing into SU with new IP address}
\item{
\begin{enumerate}
\item{If successful: set new IP setting in /etc/network/interfaces}
\item{If unsuccessful: power cycle SU and check settings starting with step 3}
\end{enumerate}}
\item{Reset SU and confirm new settings}
\end{enumerate}

\subsection*{Method 2: Workstation Setup}
The Workstation setup method requires:
\begin{itemize}
\item{HDMI Monitor and cable}
\item{{\bf Powered} USB Hub}
\item{USB Keyboard}
\item{USB Mouse}
\end{itemize}

\begin{enumerate}
\item{Connect an HDMI display, keyboard and mouse}
\item{Set network information through GUI}
\end{enumerate}

\begin{centering}
\section*{Website Setup}
\end{centering}

\subsection*{Method 3: Manual microSD Editing}

The file that need to be edited on the rootfs partition are:

\begin{verbatim}
/etc/network/interfaces
/etc/resolv.conf
/etc/ssh/sshd_config
\end{verbatim}

The interfaces file lists the IP information of the machine whole the resolv.conf file is for the DNS information.
The sshd\_config file on line 13 sets the port with which SSH is allowed.

\subsection*{Starting apache2}

To get apache2 running only one change needs to be made in the /etc/apache2/httpd.conf file.

\begin{verbatim}
Line 96:	#ServerName www.example.com:80
\end{verbatim}

Needs to be uncommented and changed to the hostname of the computer, e.g.:

\begin{verbatim}
Line96:	ServerName bobholz-3.ess.washington.edu:80
\end{verbatim}

Then httpd needs to be restarted:

\begin{verbatim}
sudo httpd -k restart
\end{verbatim}

\subsection*{Setting up the website}

All changes to the website need to be made in the /home/sferix/public\_html\_static folder, this folder is copied to /home/sferix/public\_html during start up. Changes to public\_html are not saves as the folder is located in system RAM due to SD card read/write limitations. A restart in not necessary if the public\_html\_static contents are copied to public\_html.






%\bibliography{/users/michael/documents/ess/research/library}
%\bibliographystyle{/users/michael/documents/ess/research/agu}

\end{document}